\chapter{Project Plan}

\section{Overview}
The \sys{} project used a distributed development model as suggested by the
\texttt{git}~\cite{git:website} version control system. Each team member was expected
to keep up with overall development through automated emails sent out via \texttt{git}
hook scripts run on every successful push.

Initial planning and design of the \sys{} language came out of the previous experience of
group members in image processing. Specification of language syntax was driven by the desire
to simplify basic image processing and linear algebra calculations which are onerous to
implement in other languages. The core of this development hinged on the \emph{convolution}
(section~\ref{ssec:convolutionop}) operator. When used in combination with a flexible matrix
definition syntax, a simple convolution operator can eliminate quadruply--nested for--loops and
clarify program operation. This paradigm lead us directly to our elegant matrix definition
syntax, and a \texttt{Kernel} (section~\ref{ssec:kernels}) formalism that allows efficient
ordering of calculations and further reduces the number of lines of code necessary to implement
most algorithm.

We took a practical approach to formalizing the language semantics by attempting to implement
a real--world algorithm in the most compact yet readable manner possible. We chose
the \emph{Sobel}~\cite{sobel:wikipedia} operator which is a standard edge detection algorithm
used in computer vision. Calculating the Sobel operator involves a luminance calculation,
two separate convolutions, and two additional calculations on the convolution output. Using
this algorithm as a guide, we developed a syntax which balanced complexity and readability,
and is able to implement the complete Sobel operator in 12 lines of code
(see section~\ref{sec:tutorial:fullexample} for a source listing).

After language specification and semantics were reasonably well-defined, we began development
of the scanner and parser. We quickly realized that the most challenging pieces of the \sys{}
design would be the strict type--checking and backend C implementations. With this in mind,
the project was put on a loose timeline where priority was given to the verifier
(see section~\ref{chap:archdesign}) and to the C library implementation of the \sys{} syntax.

Fortunately formal verification of types and syntax could happen substantially in parallel
with the backend and semantically checked abstract syntax tree (SAST) generation. However, we found
that the details of the generated C code, the backend implementation, and the SAST were
inexorably entwined. Thus the major focus of development at the end of the project was on
this area, and collaborative programming techniques such as pair-programming were employed to
speed development and improve code quality.

%Identify process used for planning, specification, development and testing

\section{Administration}
Show your project timeline

Identify roles and responsibilities of each team member

\section{Coding Style Guide}
Include a one-page programming style guide used by the team

\section{Development Environment}
Describe the software development environment used (tools and languages)

A complete log of project activity can be found in appendix~\ref{appendix:vcshistory}
