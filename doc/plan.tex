\chapter{Project Plan}
\label{chap:projplan}

\section{Overview}
The \sys{} project used a distributed development model as suggested by the
\texttt{git}~\cite{git:website} version control system. Each team member was expected
to keep up with overall development through automated emails sent out via \texttt{git}
hook scripts run on every successful push.

Initial planning and design of the \sys{} language came out of the previous experience of
group members in image processing. Specification of language syntax was driven by the desire
to simplify basic image processing and linear algebra calculations which are onerous to
implement in other languages. The core of this development hinged on the \emph{convolution}
(section~\ref{ssec:convolutionop}) operator. When used in combination with a flexible matrix
definition syntax, a simple convolution operator can eliminate quadruply--nested for--loops and
clarify program operation. This paradigm lead us directly to our elegant matrix definition
syntax, and a \texttt{Kernel} (section~\ref{ssec:kernels}) formalism that allows efficient
ordering of calculations and further reduces the number of lines of code necessary to implement
most algorithm.

We took a practical approach to formalizing the language semantics by attempting to implement
a real--world algorithm in the most compact yet readable manner possible. We chose
the \emph{Sobel}~\cite{sobel:wikipedia} operator which is a standard edge detection algorithm
used in computer vision. Calculating the Sobel operator involves a luminance calculation,
two separate convolutions, and two additional calculations on the convolution output. Using
this algorithm as a guide, we developed a syntax which balanced complexity and readability,
and is able to implement the complete Sobel operator in 12 lines of code
(see section~\ref{sec:tutorial:fullexample} for a source listing).

After language specification and semantics were reasonably well-defined, we began development
of the scanner and parser. We quickly realized that the most challenging pieces of the \sys{}
design would be the strict type--checking and backend C implementations. With this in mind,
the project was put on a loose timeline where priority was given to the verifier
(see section~\ref{chap:archdesign}) and to the C library implementation of the \sys{} syntax.

Fortunately formal verification of types and syntax could happen substantially in parallel
with the backend and semantically checked abstract syntax tree (SAST) generation. However, we found
that the details of the generated C code, the backend implementation, and the SAST were
inexorably entwined. Thus the major focus of development at the end of the project was on
this area, and collaborative programming techniques such as pair-programming were employed to
speed development and improve code quality.
\\ % FORMAT HACK!

Our testing procedure, described in chapter~\ref{chap:testplan}, is based around unit tests
divided into four major categories: syntactic tests, semantic/type tests, CString tests, and
functional output tests. Each category is designed to exercise a specific area of the \sys{}
compiler, and each test uses a simple pass/fail metric. Unit tests were developed in parallel
to other development, and were used extensively to fix bugs and refine the language syntax.

\section{Administration}

Although \sys{} development did not follow a strict timeline, there was a well-defined order
in which major pieces of the design had to be completed.
Figure~\ref{fig:timeline} shows the timeline that \sys{} development followed which was
reconstructed from the version control logs. Note that there was a compression of activity
near the end of the development cycle. This was an unfortunate side-effect of a loosely-defined
schedule.
\begin{figure}[h!]
\begin{center}
  \begin{tikzpicture}[x=0.75cm, y=0.9cm]
    \begin{ganttchart}[vgrid, hgrid,
                       bar={draw=green!90, fill=green!20, rounded corners=3pt},
                       milestone={fill=orange!20, draw=orange!90}]{16}
      \gantttitle{Sept.}{4} \gantttitle{Oct.}{4} \gantttitle{Nov.}{5} \gantttitle{Dec.}{3} \\
      \ganttbar{Language Definition}{2}{3.5} \\
      \ganttmilestone{Language Proposal}{3.5} \\
      \ganttbar{Framework/Infrastructure}{3.5}{6.25} \\
      \ganttmilestone{LRM}{8.25} \\
      \ganttbar{Scanner/Parser}{5}{10.25} \\
      \ganttbar{Verifier}{9}{13.5} \\
      \ganttbar{SAST}{9}{14} \\
      \ganttbar{Semantic/Backend}{13}{15} \\
      \ganttbar{C Implementation}{13.5}{15} \\
      \ganttbar{Unit Tests/Bugfixes}{12.5}{15} \\
      \ganttmilestone{Functional \sys{}}{14.75}
    \end{ganttchart}
  \end{tikzpicture}
  \caption{\sys{} Development Timeline}
  \label{fig:timeline}
\end{center}
\end{figure}

Each member of the \sys{} team played a role in the development process. Figure~\ref{fig:devroles}
details the code and documentation responsibilities of each member. In addition to each documentation
area specified in the figure, each team member contributed a \emph{lessons learned} paragraph. These
were collected into Chapter~\ref{chap:lessons}.

\begin{figure}[h!]
  \begin{center}
    \begin{tabular}{ l | l | l }
      {\bf Developer} & {\bf Code Responsibilities} & {\bf Documentation Responsibilities} \\
      \hline
      \multirow{3}{*}{Jeremy Andrus} & Framework, Scanner/Parser, & Introduction (Ch.\ref{chap:intro}), LRM (Ch.\ref{chap:lrm}), \\
      & Verifier, C Implementation & Project Plan (Ch.\ref{chap:projplan}), Code Reference (App.\ref{chap:coderef}), \\
      & & VCS History (App.\ref{chap:vcshistory}) \\
      \hline
      \multirow{2}{*}{Robert Martin} & SAST, Semantic, Backend & Architectural Design (Ch.\ref{chap:archdesign}), LRM (Ch.\ref{chap:lrm}), \\
      & & Introduction (Ch.\ref{chap:intro})\\
      \hline
      \multirow{2}{*}{Kevin Sun} & Unit Tests, Verifier Bugfixes & Language Tutorial (Ch.\ref{chap:tutorial}), Test Plan (Ch.\ref{chap:testplan}),\\
      & & LRM (Ch.\ref{chap:lrm})\\
      \hline
      \multirow{2}{*}{Yongxu Zhang} & Unit Tests, Verifier Bugfixes & Test Plan (Ch.\ref{chap:testplan}) \\
      & & \\
    \end{tabular}
    \caption{Roles and Responsibilities}
    \label{fig:devroles}
  \end{center}
\end{figure}

\section{Development Environment}
The \sys{} compiler is written in Objective Caml (OCaml) and C/C++. The bulk of the compilation is
done in OCaml which generates C/C++. A \sys{} C library was written which implements the
low-level details of the image processing described by the language including structures,
memory management, and a convolution framework written in a C++ template function.

Compiling \sys{} requires the \texttt{gcc} compiler and the OCaml toolchain available for
download from the OCaml website: \url{http://caml.inria.fr/download}. In order to produce
machine-executable binaries, \sys{} must invoke the \texttt{g++} compiler which should be
accessible from the user's default \emph{PATH}.

\sys{} documentation, including all charts, code examples, and course-related documents,
were written in \LaTeX{}. This allowed the team to work on the same document in a coherent,
distributed manner.

All source code and documentation was managed in the \texttt{git}~\cite{git:website}
distributed version control system. A ``master'' repository was kept on a Columbia CRF backed-up
server, and \emph{hook} scripts were used to notify all team members when code was pushed to this
repository. A complete log of project activity can be found in Appendix~\ref{chap:vcshistory}.
