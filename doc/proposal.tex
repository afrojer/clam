\section*{Language Proposal}

\definecolor{DarkBlue}{rgb}{0.0, 0.08, 0.65}
\definecolor{DarkRed}{rgb}{0.65, 0.08, 0.0}
\definecolor{DarkGreen}{rgb}{0.08, 0.35, 0.08}
\definecolor{Grey}{rgb}{0.55, 0.55, 0.55}
\definecolor{Yellow}{cmyk}{0.0, 0.08, 0.5, 0.3}
\definecolor{Brown}{rgb}{0.55, 0.27, 0.08}
\definecolor{LightBlue}{cmyk}{0.8, 0.1, 0.0, 0.1}
\lstdefinelanguage{PLTF11}{
    classoffset=0,
	morekeywords={somethingsomethingsomething},
    keywordstyle={\color{Yellow}\itshape\bfseries},
    classoffset=1,
    morekeywords={Uint8,Uint16,Uint32,Int8,Int16,Int32,Angle,Matrix,Image,Kernel},
    keywordstyle={\color{DarkBlue}\bfseries},
    classoffset=2,
	morekeywords={imgread,imgwrite},
	keywordstyle={\color{Brown}},
	otherkeywords={:=,**,=,|,:},
	moredelim=*[s][\color{LightBlue}]{\#[}{]},
	moredelim=*[s][\color{DarkGreen}\itshape]{\$(}{)},
    morecomment=[l]{//},
    commentstyle={\color{Grey}\bfseries},
    morestring=[b]",
    stringstyle={\color{DarkRed}},
    sensitive=true,
}

\sys{} is a linear algebra manipulation language specifically targeted for
image processing. It provides an efficient way to express complex image
manipulation algorithms through compact matrix operations. Traditional image
processing is performed using a language such as C, or C++. Algorithms in these
languages are quite complex and error-prone due to the large number of lines of
code required to implement something as conceptually simple as, "make this image
blurry." The complexity arises from the need to perform elaborate calculations
on every pixel in an image. For example, to blur an image you first need to
calculate the luminance of the pixel (from the red, green, and blue channels),
then you need to mathematically combine this with the luminance of adjacent pixels,
and finally re-calculate red, green, and blue values for an output image.

\sys{} will simplify image processing, and more generally linear algebra, through
domain-specific data types and operators. The basic data type in \sys{} is a 
\texttt{Matrix}. Matrices can be manipulated by operators that perform functions
such as matrix multiplication, or rotation. An \texttt{Image} is another \sys{}
data type which is expressed as a collection of matrices, or channels. For example,
when reading an image into memory, \sys{} creates a \emph{Red}, \emph{Green}, and
\emph{Blue} channel automatically. Additional Image channels can either be assigned,
or calculated using an expression syntax which defines a calculation involving the
values of other, previously defined, channels. The basic image processing operator
in \sys{} is the convolution operator. This operator takes a channel and a
\texttt{Kernel}, another basic data type, and outputs an \texttt{Image}. This
operator convolves each \texttt{Matrix} within the \texttt{Kernel} with the input channel,
and collects the resulting output channels into an \texttt{Image}.

Two primary use cases of \sys{} are basic image information extraction, and
filtering. The compact syntax and powerful basic data types of \sys{} will make
information extraction, such as finding all the edges in an image, simple, compact,
and easy to read.

\subsection*{Features}
\sys{} uses implicit loops, i.e. there is no explicit looping construct in the
language. Loops are implicitly defined by per-pixel matrix or convolution
operations. Additionally, \sys{} automatically determines or calculates image and matrix
dimensions. There is no need to explicitly size these data types. This further
reduces complexity, and eliminates frequent mistakes such as going beyond array
bounds in a calculation.

\subsection*{Example Syntax}

The goal of the \sys{} syntax will be to make conceptually simple image
manipulations into simple language constructs. For instance, convolutions make
frequent use of constant matrices, so our language will provide a simple way
to specify them, such as:
\begin{lstlisting}[language=PLTF11]
    Matrix sobelGy := { +1 +2 +1 |  0  0  0 | -1 -2 -1 };
\end{lstlisting}

Another common image processing technique is performing the same calculation
on every pixel in the image. An example of this is calculating the luminance
of a pixel from the red, green, and blue channels. \sys{} makes this calculation
simple and compact by defining an additional image channel. Assuming there exists
an instance of a \texttt{Image} variable named, \emph{myimg}, a channel can be
added to the image with:
\begin{lstlisting}[language=PLTF11]
    Int32 myimg:Luminance := #[ (3*Red + 6*Green + 1*Blue) / 10 ];
\end{lstlisting}
Where the expression within \texttt{\#[ \ldots ]} is evaluated once for every
pixel in the \texttt{Image}. The \emph{Red}, \emph{Green}, and \emph{Blue} variables
correspond to previously defined channels in \emph{myimg}, and their values during
expression evaluation will be the value of the corresponding channel at the current
pixel location.

Image processing also frequently involves describing a series of operations that
should be carried out for each pixel, and then repeating it for every pixel in an
image. \sys{} makes it simple to describe this process through the \texttt{Kernel}
data type and the convolution operator. Here is an example of how one might perform
a Sobel edge detector in the \sys{} language:
\lstinputlisting[language=PLTF11,numbers=left,frame=single]{src/sobel.imp}
