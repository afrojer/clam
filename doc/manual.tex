\newcommand{\startsyn}{\begin{center}\begin{tabular}{l}}
\newcommand{\stopsyn}{\end{tabular}\end{center}}

\section{Introduction}
The \sys{} programming language is a linear algebra manipulation language
specifically targeted for image processing. It provides an efficient way to
express complex image manipulation algorithms through compact matrix
operations. \sys{} programs are first compiled into a ''C`` module which is
further compiled into a machine binary by an existing C compiler. This two-step
process is completely automated by the \sys{} compiler, and by default no C code
is output (this can be changed with compiler arguments - see section~\ref{sec:compiling}).

This language reference is inspired by the C reference manual~\cite{DBLP:KernighanR88}.
It details the syntax of the \sys{} language.

\section{Lexical Conventions}
\label{sec:lex}

\subsection{Tokens}
\label{ssec:tokens}
The tokens in \sys{} are broken down as follows: We have reserved
keywords, identifiers, constants, control characters, and operators.
The end of a token is defined by the presence of a newline, space,
or tab character (whitespace), or by the presence of a character that
cannot possibly be part of the current token.

\subsection{Comments}
\label{ssec:comments}
Comments are demarcated with an opening /* and closing */, as in C.
Any characters inside the comment boundaries are ignored. Comments
can be nested.

\subsection{Keywords}
\label{ssec:keywords}
The reserved keywords in \sys{} are:
\begin{center}\begin{tabular}{l l l l}
Image & Import & Int8 & Uint8\\
Kernel & Export & Int16 & Uint16\\
Channel & Angle & Int32 & Uint32\\
Calc & Float & &
\end{tabular}\end{center}

\subsection{Identifiers}
\label{ssec:identifiers}
Identifiers are composed of an upper or lower-case letter immediately
followed by any number of additional letters and/or digits. Identifiers
are case sensitive, so ``foo'' and ``Foo'' are different identifiers.
Identifiers cannot be keywords, and underscores are disallowed.

\subsection{Constants}
\label{ssec:constants}
In \sys{} there are 5 types of constants: string constants, matrix
constants, integers, angles, and floats.

\subsubsection{Numeric Constants}
\label{sssec:numericconstants}

\emph{Integers} are repesented by a series of number characters.
% the negative sign is dealt with by the unary '-' operator

Angles are represented by a series of number characters with an
optional period character, followed by a lower-case ``a''.

Floats are represented by a series of number characters with an
optional period character, followed by a lower-case ``f''.

\subsubsection{Matrix Constants}
\label{sssec:matrixconstants}
Matrix constants are represented by an opening curly brace, followed
by a series of \emph{numeric-expressions} separated by whitespace or
comma characters. The comma  characters represents the division between
the rows of the matrix. Each row must have the same number of
\emph{numeric-expressions}, but the matrix need not be square.

A matrix constant may also have an optional fraction preceding it,
which indicates that every value in the matrix should be multiplied
by that fraction. The fraction will be expressed as an opening
bracket character, a \emph{numeric-expression} representing the
numerator, a forward-slash character, a \emph{numeric-expression}
representing the denominator, and a closing bracket character.

\startsyn
\texttt{\{} \emph{numeric-expr} \emph{numeric-expr} \ldots \texttt{,} \emph{numeric-expr} \emph{numeric-expr}\ldots \texttt{\}} \\
\texttt{[}\emph{numeric-expr} \texttt{/} \emph{numeric-expr} \texttt{]}\texttt{\{} \emph{numeric-expr} \emph{numeric-expr} \ldots \texttt{,} \emph{numeric-expr} \emph{numeric-expr}\ldots \texttt{\}}
\stopsyn

The following is an example of a matrix constant.
\begin{lstlisting}[language=CLAM,escapechar=\%]
    Matrix sobelGy := [1 / 9]{1 3 1 , 2 -5 2 , 1 3 1 }
\end{lstlisting}

\subsection{String Literals}
\label{ssec:strings}

String constants are demarcated by double quote characters or single
quote characters. Consecutive string constants will be automatically
appended together into a single string constant.
\startsyn
\texttt{"}\emph{string-constant}\texttt{"}
\stopsyn

\section{Meaning of Identifiers}
\label{sec:identmeaning}

\subsection{Basic Types}
\label{ssec:types}
There are four basic types defined by the \sys{} language.
Type identifiers always begin with an upper-case letter followed by a sequence
of zero or more legal identifier characters. The list of built-in types is as follows:
\startsyn
\texttt{Channel} \\
\texttt{Image} \\
\texttt{Calc} \\
\texttt{Kernel}
\stopsyn

\subsubsection{Atom Types}
\label{sssec:atomtypes}
The \texttt{Channel} and \texttt{Calc} built-in types may be further modified
to specify individual element, or ``atom'' types. This specifies either the type
of each element of the matrix which makes up the \texttt{Channel}, or the type
of the resulting calculation performed by a \texttt{Calc} object.
An \emph{atom-type} identifier is denoted using the \texttt{<} and \texttt{>}
characters immediately following the identifier of the object whose atom type
is being specified:
\startsyn
\emph{basic-type} \emph{identifier}\texttt{<}\emph{atom-type}\texttt{>}
\stopsyn
Legal \emph{atom-type}s are as follows:
\startsyn
Uint8 \\
Uint16 \\
Uint32 \\
Int8 \\
Int16 \\
Int32 \\
Angle
\stopsyn

\subsection{Type Qualifiers}
\label{ssec:typequal}
Discuss the use of the \texttt{@} symbol to denote a special matrix which when used
in calculation does not produce in a discrete channel in the result.

Discuss the use of another symbol which tells the matrix composition that calculation must finish before proceeding (i.e. it actively inhibits parallel execution of convolution - probably useful if a subsequent calculation requires a neighborhood of previously calculated values).


\section{Definitions and Objects}
\label{sec:defobj}
Here we should discuss the difference between ``defining'' a matrix (e.g. via calculation in escaped-C syntax) and actual object assignment performed by computation and assignment operators\ldots


\section{Expressions}
\label{sec:expressions}

\subsection{Primary Expressions}
\label{ssec:primaryexpresions}
identifiers, constants, strings. The type of the expressions depends on the identifier, constant or string.

\subsection{Unary Operators}
\label{ssec:unaryoperators}
There are two unary operators in \sys{}, and they are only used with a
numeric-valued operand.
These expressions are grouped right-to-left:
\startsyn
\texttt{+}\emph{numeric-expression} \\
\texttt{-}\emph{numeric-expression}
\stopsyn

\subsubsection{\texttt{+} operator}
This operator forces the value of its numeric operand to be positive.
The resulting expression is of numeric type with a value equal to the
absolute value of the numeric operand.

\subsubsection{\texttt{-} operator}
This operator forces the value of its numeric operand to be negative.
The resulting expression is of numeric type with a value equal to the
negative of the numeric operand.

\subsection{Channel Expresions}
\label{ssec:channelexpressions}
Channels for the basis of both \texttt{Image} and \texttt{Kernel} types, and
\sys{} has several operators which manipulate channels.

\subsubsection{\texttt{:} operator}
\label{sssec:colonop}
Extract or use an individual channel in an image.
\startsyn
\emph{image-identifier}\texttt{:}\emph{channel-identifier}
\stopsyn
The resulting expression has the \texttt{Matrix} type corresponding to the
extracted channel.

\subsubsection{\texttt{\$()} operator}
\label{sssec:evalop}
This operator forces the evaluation of a previously defined Image channel. It
is generally used in the context of a convolution operation.
\startsyn
\texttt{\$(}\emph{channel-expression}\texttt{)}
\stopsyn
The resulting expression is of \texttt{Channel} type.

\subsection{Channel Composition Operators}
\label{ssec:channelops}
These operators compose an \texttt{Image} from one or more \texttt{Channels}.
All channel composition operators are left-to-right associative.

\subsubsection{\texttt{|} operator}
\label{sssec:barop}
Compose two (or more) \texttt{Channel}s. The resulting expression is a
\emph{multi-channel-expression}, and can be assigned to either an \texttt{Image}
or a \texttt{Kernel} object.
\startsyn
\emph{channel-expression} \texttt{|} \emph{channel-expression} \\
\emph{multi-channel-expression} \texttt{|} \emph{channel-expression}
\stopsyn
Note that \texttt{Channel}s are appended in order, and subsequent operations
may rely on this order.

\subsubsection{\texttt{||} operator}
\label{sssec:doublebarop}
Compose two (or more) \texttt{Channel}s. The resulting expression is a
\emph{multi-channel-expression}, and can be assigned to either an \texttt{Image}
or a \texttt{Kernel} object. This operator differs from the \texttt{|} operator in that
it forces the serial computation of channel values. This allows subsequent channel
value calculations to use neighboring pixels of previously calculated channels.
\startsyn
\emph{channel-expression} \texttt{||} \emph{channel-expression} \\
\emph{multi-channel-expression} \texttt{||} \emph{channel-expression}
\stopsyn

\subsection{Escaped ``C'' Expression}
\label{ssec:escapedC}
Talk about the \texttt{\#[\ldots]\#} operator.

\subsection{Channel Composition Expresions}
%Talk about the $|$, $|=$, and $||$ expressions.

\subsection{Assignmet Expresions}
\label{ssec:assignment}

\subsubsection{\texttt{|=} assignment operator}

\subsubsection{\texttt{\^=} assignment operator}

\subsubsection{\texttt{=} assignment}


\section{Statements}
\label{sec:statements}

Statements in \sys{} always end in a semi-colon. No statement
can return a value. All statements should either declare a variable,
define or modify the definition of a variable, execute
some calculation based on previously declared variables with
the result stored in previously declared variables, or write an image to a file.
"Statements" consisting of a lone r-value (an identifier,
channel reference, escaped-C string, matrix, or \texttt{imgread()} call)
will be accepted but will perform no useful action - they are evaluated but
their return values are discarded (or are they erased during translation?)

\section{Program Definition}
A program in the \sys{} language is simply a sequence of statements which
are executed in order.

\section{Scope Rules}
All identifiers in the \sys{} language are global.

\section{Declarations}
All variables must be declared before they can be used.
Variables become usable after the end of the semi-colon of the statement
in which its contained.
\section{Grammar}

\subsection{Expressions}

\begin{center}\begin{tabular}{l l}
\emph{expression}: & \emph{identifier}\\
& \emph{integer}\\
& \emph{literal-string}\\
& \emph{c-string}\\
& \emph{matrix}\\
& \emph{matrix-scale matrix}\\
& \emph{kernel-calc-list}\\
& \emph{channel-ref}\\
& \emph{identifier} \texttt{=} \emph{expression}\\
& \emph{channel-ref} \texttt{=} \emph{expression}\\
& \emph{channel-ref} \texttt{**} \emph{identifier}\\
& \emph{identifier} \texttt{|=} \emph{expression}\\
& \emph{library-function} \texttt{(} \emph{argument-list} \texttt{)}\\
\\
\emph{matrix-scale}: & \texttt{[} \emph{integer} \texttt{/} \emph{integer} \texttt{]}\\
\\
\emph{matrix}: & \texttt{\{} \emph{row-list} \texttt{\}}\\
& \emph{matrix-scale} \texttt{\{} \emph{row-list} \texttt{\}}\\
\\
\emph{row-list}: & \emph{matrix-row}\\
& \emph{row-list} \texttt{,} \emph{matrix-row}\\
\\
\emph{matrix-row}: & \emph{integer}\\
& \emph{matrix-row} \emph{integer}\\
\\
\emph{kernel-calc-list}: & \texttt{@}\emph{identifier}\\
& \texttt{|} \emph{identifier}\\
& \texttt{| @}\emph{identifier}\\
& \emph{kernel-calc-list} \texttt{|} \emph{identifier}\\
& \emph{kernel-calc-list} \texttt{| @}\emph{identifier}\\
\\
\emph{channel-ref}: & \emph{identifier}\texttt{:}\emph{identifier}\\
\\
\emph{argument-list}: & \emph{literal-string}\\
& \emph{argument-list} \texttt{,} \emph{literal-string}\\
\\
\emph{library-function}: & \texttt{imgread}\\
& \texttt{imgwrite}\\
\end{tabular}\end{center}

\subsection{Declarations}

\begin{center}\begin{tabular}{l l}
\emph{declaration}: & \texttt{Image} \emph{identifier}\\
& \texttt{Kernel} \emph{identifier}\\
& \texttt{Calc} \emph{identifier}\\
& \texttt{Calc} \emph{identifier}\texttt{<}\emph{atomic-type}\texttt{>}\\
\\
\emph{atomic-type} & \texttt{Uint8}\\
& \texttt{Uint16}\\
& \texttt{Uint32}\\
& \texttt{Int8}\\
& \texttt{Int16}\\
& \texttt{Int32}\\
& \texttt{Angle}\\
\end{tabular}\end{center}


\subsection{Statements}

\begin{center}\begin{tabular}{l l}
\emph{statement}: & \emph{expression} \texttt{;}\\
& \emph{declaration} \texttt{;}\\
& \emph{declaration} \texttt{=} \emph{expression} \texttt{;}\\
& \emph{declaration} \texttt{:=} \emph{expression} \texttt{;}\\
\\
\emph{program}: & \emph{statement}\\
& \emph{statement} \emph{program}\\
\end{tabular}\end{center}




\clearpage
\section{Examples}
The following example implements a Sobel image filter using the
\sys{} language.
\lstinputlisting[language=CLAM,numbers=left,frame=single]{src/sobel.clam}
