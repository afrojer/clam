\newcommand{\startsyn}{\begin{center}\begin{tabular}{l}}
\newcommand{\stopsyn}{\end{tabular}\end{center}}

\section{Introduction}
The \sys{} programming language is a linear algebra manipulation language
specifically targeted for image processing. It provides an efficient way to
express complex image manipulation algorithms through compact matrix
operations. \sys{} programs are first compiled into a ''C`` module which is
further compiled into a machine binary by an existing C compiler. This two-step
process is completely automated by the \sys{} compiler, and by default no C code
is output (this can be changed with compiler arguments - see section~\ref{sec:compiling}).

This language reference is inspired by the C reference manual~\cite{DBLP:KernighanR88}.
It details the syntax of the \sys{} language.

\section{Lexical Conventions}
\label{sec:lex}

\subsection{Tokens}
\label{ssec:tokens}
The tokens in \sys{} are broken down as follows: We have reserved
keywords, constants, control characters, identifiers, and operators.
The end of a token is defined by the presence of a newline, space,
or tab character (whitespace), or by the presence of a character that
cannot possibly be part of the current token.

\subsection{Comments}
\label{ssec:comments}
Comments are demarcated with an opening /* and closing */, as in C.
Any characters inside the comment boundaries are ignored. Comments
can be nested.

\subsection{Identifiers}
\label{ssec:identifiers}
Identifiers are composed of an upper or lower-case letter immediately
followed by any number of additional letters and/or digits. Identifiers
are case sensitive, so ''foo`` and ''Foo`` are different identifiers.
Identifiers cannot be keyworeds, and underscores are disallowed.

\subsection{Keywords}
\label{ssec:keywords}
The reserved keywords in \sys{} are:
Import, Export, Kernel, Channel, 
Int8, Int32, Angle, and Float.


\subsection{Constants}
\label{ssec:constants}
In \sys{} there are 5 types of constants: string constants, matrix
constants, integers, angles, and floats.

\subsubsection{Numeric Constants}
\label{sssec:numericconstants}

Integers are repesented by a series of number characters, with an
optional negative sign in front.

Angles are represented by a series of number characters with an
optional period character, followed by a lower-case ''a``.

Floats are represented by a series of number characters with an
optional period character, followed by a lower-case ''f``.

\subsubsection{Matrix Constants}
\label{sssec:matrixconstants}
Matrix constants are represented by an opening curly brace, followed
by a series of integers separated by whitespace or pipe characters.
The pipe characters represent the division between the rows of the
matrix. Each row must have the same number of integers, but the
matrix need not be square.

A matrix constant may also have an optional fraction preceding it,
which indicates that every value in the matrix should be multiplied
by that fraction. The fraction will be expressed as an opening
bracket character, an integer representing the numerator, a pipe
character, an integer representing the denominator, and a closing
bracket character.

The following is an example of a matrix constant.
\begin{lstlisting}[language=CLAM,escapechar=\%]
    Matrix sobelGy := [N | D]{R1C1 R1C2%\ldots% | R2C1 R2C2%\ldots% | R3C1 R3C2%\ldots%};
\end{lstlisting}

\subsection{String Literals}
\label{ssec:strings}

String constants are demarcated by double quote characters or single
quote characters. Consecutive string constants will be automatically
appended together into a single string constant.


\section{Meaning of Identifiers}
\label{sec:identmeaning}

\subsection{Basic Types}
\label{ssec:types}
There are four basic types defined by the \sys{} language.
Type identifiers always begin with an upper-case letter followed by a sequence
of zero or more legal identifier characters. The list of built-in types is as follows:
\startsyn
\texttt{Image} \\
\texttt{Calc} \\
\texttt{Kernel}
\stopsyn 

\subsubsection{Atom Types}
\label{sssec:atomtypes}
The \texttt{Calc} type may be further modified
to specify individual element, or ``atom'' types. This specifies either the type
of the resulting calculation performed by a \texttt{Calc} object. (When calculation
results go out of bounds for their type, their values are clamped - set to the max or min value.)
An \emph{atom-type} identifier is denoted using the \texttt{<} and \texttt{>}
characters immediately following the identifier of the object whose atom type
is being specified:
\startsyn
\emph{basic-type} \emph{identifier}\texttt{<}\emph{atom-type}\texttt{>}
\stopsyn
Legal \emph{atom-type}s are as follows:
\startsyn
Uint8 \\
Uint16 \\
Uint32 \\
Int8 \\
Int16 \\
Int32 \\
Angle
\stopsyn
In the absence of an atom-type specification, the atom-type defaults to Uint8,
which implies a range of integers from 0-255 inclusive. (This is also the atomic-type
for the default \emph{Red}, \emph{Green} and \emph{Blue} channels.)


\section{Objects and Definitions}
\label{sec:objdef}
An \emph{object} in \sys{} is either a named collection of \texttt{Channel}s, called an
\texttt{Image}, or a named collection of calculation bases, called a
\texttt{Kernel}. A \texttt{Channel} is a mathematical matrix of numeric values
whose individual components are not directly accessible via \sys{} language
semantics -- \texttt{Channel} values are manipulated via the convolution
operator (see~\ref{ssec:convolutionop}). A calculation basis, known as a
\texttt{Calc}, is a collection of either calculation constants
(see~\ref{sssec:calcconstants}) or calculation expressions (see~\ref{ssec:escapedC}),
or both.

\subsection{\texttt{Image} objects}
\label{ssec:images}
An \texttt{Image} is a collection of named \texttt{Channel}s. \texttt{Channel}s can
be dynamically added \comment{or removed} using the channel composition
operator (see section~\ref{sssec:barequalop}, or by assigning to a previously
undeclared \texttt{Channel} name. 

For example, to create a gray-scale image from a single, pre-existing
\texttt{Channel}:
\begin{lstlisting}[language=CLAM,escapechar=\%]
Image outImg;
outImg:Red = calcImg:G;
outImg:Green = calcImg:G;
outImg:Blue = calcImg:G;
\end{lstlisting}

\subsection{\texttt{Kernel} objects}
\label{ssec:kernels}
A \texttt{Kernel} is an ordered collection of calculation bases which is used by the convolution
operator (see section~\ref{ssec:convolutionop}). Each calculation has to be identified with a \texttt{Calc}
identifier, but the underlying basis can be either
a matrix calculation constant (see~\ref{sssec:calcconstants}) or a escaped C expression
(see~\ref{ssec:escapedC}). A \texttt{Kernel} is composed either using the composition
operator (see section~\ref{sssec:barop}), or the \texttt{|=} assignment operator (see section~\ref{sssec:barequalop}).

To see how a \texttt{Kernel} is used in calculation, see section~\ref{ssec:convolutionop}.

\section{Expressions}
\label{sec:expressions}

\subsection{Primary Expressions}
\label{ssec:primaryexpresions}
identifiers, constants, strings. The type of the expressions depends on the identifier, constant or string.

\subsection{Unary Operators}
\label{ssec:unaryoperators}
There are two unary operators in \sys{}, and they are only used with a
numeric-valued operand.
These expressions are grouped right-to-left:
\startsyn
\texttt{+}\emph{numeric-expression} \\
\texttt{-}\emph{numeric-expression}
\stopsyn

\subsubsection{\texttt{+} operator}
This operator forces the value of its numeric operand to be positive.
The resulting expression is of numeric type with a value equal to the
absolute value of the numeric operand.

\subsubsection{\texttt{-} operator}
This operator forces the value of its numeric operand to be negative.
The resulting expression is of numeric type with a value equal to the
negative of the numeric operand.

\subsection{Channel Expresions}
\label{ssec:channelexpressions}
Channels for the basis of both \texttt{Image} and \texttt{Kernel} types, and
\sys{} has several operators which manipulate channels.

\subsubsection{\texttt{:} operator}
\label{sssec:colonop}
Extract or use an individual channel in an image.
\startsyn
\emph{image-identifier}\texttt{:}\emph{channel-identifier}
\stopsyn
The resulting expression has the \texttt{Matrix} type corresponding to the
extracted channel.

\subsubsection{\texttt{\$()} operator}
\label{sssec:evalop}
This operator forces the evaluation of a previously defined Image channel. It
is generally used in the context of a convolution operation.
\startsyn
\texttt{\$(}\emph{channel-expression}\texttt{)}
\stopsyn
The resulting expression is of \texttt{Channel} type.

\subsection{Channel Composition Operators}
\label{ssec:channelops}
These operators compose an \texttt{Image} from one or more \texttt{Channels}.
All channel composition operators are left-to-right associative.

\subsubsection{\texttt{|} operator}
\label{sssec:barop}
Compose two (or more) \texttt{Channel}s. The resulting expression is a
\emph{multi-channel-expression}, and can be assigned to either an \texttt{Image}
or a \texttt{Kernel} object.
\startsyn
\emph{channel-expression} \texttt{|} \emph{channel-expression} \\
\emph{multi-channel-expression} \texttt{|} \emph{channel-expression}
\stopsyn
Note that \texttt{Channel}s are appended in order, and subsequent operations
may rely on this order.

\subsubsection{\texttt{||} operator}
\label{sssec:doublebarop}
Compose two (or more) \texttt{Channel}s. The resulting expression is a
\emph{multi-channel-expression}, and can be assigned to either an \texttt{Image}
or a \texttt{Kernel} object. This operator differs from the \texttt{|} operator in that
it forces the serial computation of channel values. This allows subsequent channel
value calculations to use neighboring pixels of previously calculated channels.
\startsyn
\emph{channel-expression} \texttt{||} \emph{channel-expression} \\
\emph{multi-channel-expression} \texttt{||} \emph{channel-expression}
\stopsyn

\subsection{Escaped ``C'' Expression}
\label{ssec:escapedC}
Talk about the \texttt{\#[\ldots]\#} operator.

\subsection{Channel Composition Expresions}
%Talk about the $|$, $|=$, and $||$ expressions.

\subsection{Assignmet Expresions}
\label{ssec:assignment}

\subsubsection{\texttt{|=} assignment operator}

\subsubsection{\texttt{\^=} assignment operator}

\subsubsection{\texttt{=} assignment}


\section{Statements}
\label{sec:statements}
fill this in a bit more


\section{Program Definition}
A program in the \sys{} language is simply a sequence of statements which
are executed in order.

\section{Scope Rules}
All identifiers in the \sys{} language are global.
PROBABLY should mention how the escaped ``C'' syntax works with variable scoping\ldots

\section{Declarations}
All variables must be declared before they can be used.
PROBABLY should clarify exactly when a variable/identifier is in scope (e.g. after semi-colon of declaration statement).

\section{Grammar}

\subsection{1. Expressions.}

\begin{center}\begin{tabular}{l l}
\emph{expression}: & \emph{identifier}\\
& \emph{integer}\\
& \emph{literal-string}\\
& \emph{c-string}\\
& \emph{matrix}\\
& \emph{matrix-scale matrix}\\
& \emph{kernel-calc-list}\\
& \emph{channel-ref}\\
& \emph{identifier} \texttt{=} \emph{expression}\\
& \emph{channel-ref} \texttt{=} \emph{expression}\\
& \emph{channel-ref} \texttt{**} \emph{identifier}\\
& \emph{identifier} \texttt{|=} \emph{expression}\\
& \emph{library-function} \texttt{(} \emph{argument-list} \texttt{)}\\
\\
\emph{matrix-scale}: & \texttt{[} \emph{integer} \texttt{/} \emph{integer} \texttt{]}\\
\\
\emph{matrix}: & \texttt{\{} \emph{row-list} \texttt{\}}\\
\\
\emph{row-list}: & \emph{matrix-row}\\
& \emph{row-list} \texttt{,} \emph{matrix-row}\\
\\
\emph{matrix-row}: & \emph{integer}\\
& \emph{matrix-row} \emph{integer}\\
\\
\emph{kernel-calc-list}: & \texttt{@}\emph{identifier}\\
& \texttt{|} \emph{identifier}\\
& \texttt{| @}\emph{identifier}\\
& \emph{kernel-calc-list} \texttt{|} \emph{identifier}\\
& \emph{kernel-calc-list} \texttt{| @}\emph{identifier}\\
\\
\emph{channel-ref}: & \emph{identifier}\texttt{:}\emph{identifier}\\
\\
\emph{argument-list}: & \emph{literal-string}\\
& \emph{argument-list} \texttt{,} \emph{literal-string}\\
\end{tabular}\end{center}


\clearpage
\section{Examples}

\subsection{Gaussian Blur}
Figure~\ref{fig:clamblur} shows an example \sys{} program which performs a Gaussian
blur on an input image. The input and resulting output images are shown in
Figures~\ref{fig:clamblurin} and~\ref{fig:clamblurout}.

\begin{figure}[hb!]
  \begin{center}
  \lstinputlisting[language=CLAM]{src/blur.clam}
  \caption{Gaussian Blur Implemented in \sys{}}
  \label{fig:clamblur}
  \end{center}
\end{figure}
{
  \captionsetup{justification=centering}
  \begin{figure*}[h!]
    \centering
    \subfloat[Input Image]{
      \includegraphics*[width=7.5cm]{figures/lena.png}\label{fig:clamblurin}
    }\hfill
    \subfloat[Blurred Output Image]{
      \includegraphics*[width=7.5cm]{figures/lena-blur.png}\label{fig:clamblurout}
    }\hfill
    \captionsetup{font=bf}
    \caption{Gaussian Blur \sys{} Example}
  \end{figure*}
}

\clearpage
\subsection{Image Segmentation}
Figure~\ref{fig:clamseg} shows an example \sys{} program which performs
basic image segmentation. Pixels with a luminance value greater than 200 are
displayed as red, pixels with a luminance less than or equal to 80 are displayed as blue,
and pixels in between are displayed as Green. The input and resulting output images are shown in
Figures~\ref{fig:clamsegin} and~\ref{fig:clamsegout}.

\begin{figure}[hb!]
  \begin{center}
  \lstinputlisting[language=CLAM]{src/segment.clam}
  \caption{Simple Image Segmentation Implemented in \sys{}}
  \label{fig:clamseg}
  \end{center}
\end{figure}
{
  \captionsetup{justification=centering}
  \begin{figure*}[h!]
    \centering
    \subfloat[Input Image]{
      \includegraphics*[width=7.5cm]{figures/lena.png}\label{fig:clamsegin}
    }\hfill
    \subfloat[Segmented Output Image]{
      \includegraphics*[width=7.5cm]{figures/lena-seg.png}\label{fig:clamsegout}
    }\hfill
    \captionsetup{font=bf}
    \caption{Image Segmentation \sys{} Example}
  \end{figure*}
}
