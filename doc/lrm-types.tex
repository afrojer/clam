\section{Meaning of Identifiers}
\label{sec:identmeaning}

\subsection{Basic Types}
\label{ssec:types}
There are three basic types defined by the \sys{} language.
Type identifiers always begin with an upper-case letter followed by a sequence
of zero or more legal identifier characters. The list of built-in types is as follows:
\startsyn
\texttt{Image} \\
\texttt{Calc} \\
\texttt{Kernel}
\stopsyn 

\subsubsection{Atom Types}
\label{sssec:atomtypes}
The \texttt{Calc} type may be further modified
to specify individual element, or ``atom'' types. This specifies the type
of the resulting calculation performed by a \texttt{Calc} object (either CString or Matrix --
see section~\ref{ssec:calc}). When calculation results exceed the bounds of the specified type,
values are clamped (set to the max or min value appropriately).
An \emph{atom-type} identifier is denoted using the \texttt{<} and \texttt{>}
characters immediately following the identifier of the object whose atom type
is being specified:
\startsyn
\texttt{Calc} \emph{identifier}\texttt{<}\emph{atom-type}\texttt{>}
\stopsyn
Legal \emph{atom-type}s are as follows:
\startsyn
Uint8 \\
Uint16 \\
Uint32 \\
Int8 \\
Int16 \\
Int32 \\
Angle
\stopsyn
In the absence of an atom-type specification, the atom-type defaults to Uint8,
which implies a range of integers from 0-255 inclusive. (This is also the atomic-type
for the default \emph{Red}, \emph{Green} and \emph{Blue} channels.)

