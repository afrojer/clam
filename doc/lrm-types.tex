\section{Meaning of Identifiers}
\label{sec:identmeaning}

\subsection{Basic Types}
\label{ssec:types}
There are four basic types defined by the \sys{} language.
Type identifiers always begin with an upper-case letter followed by a sequence
of zero or more legal identifier characters. The list of built-in types is as follows:
\startsyn
\texttt{Channel} \\
\texttt{Image} \\
\texttt{Calc} \\
\texttt{Kernel}
\stopsyn

\subsubsection{Atom Types}
\label{sssec:atomtypes}
The \texttt{Channel} and \texttt{Calc} built-in types may be further modified
to specify individual element, or ``atom'' types. This specifies either the type
of each element of the matrix which makes up the \texttt{Channel}, or the type
of the resulting calculation performed by a \texttt{Calc} object.
An \emph{atom-type} identifier is denoted using the \texttt{<} and \texttt{>}
characters immediately following the identifier of the object whose atom type
is being specified:
\startsyn
\emph{basic-type} \emph{identifier}\texttt{<}\emph{atom-type}\texttt{>}
\stopsyn
Legal \emph{atom-type}s are as follows:
\startsyn
Uint8 \\
Uint16 \\
Uint32 \\
Int8 \\
Int16 \\
Int32 \\
Angle
\stopsyn

\subsection{Type Qualifiers}
\label{ssec:typequal}
Discuss the use of the \texttt{@} symbol to denote a special matrix which when used
in calculation does not produce in a discrete channel in the result.

Discuss the use of another symbol which tells the matrix composition that calculation must finish before proceeding (i.e. it actively inhibits parallel execution of convolution - probably useful if a subsequent calculation requires a neighborhood of previously calculated values).

