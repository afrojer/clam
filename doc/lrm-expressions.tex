\section{Expressions}
\label{sec:expressions}

\subsection{Primary Expressions}
\label{ssec:primaryexpresions}
identifiers, constants, strings. The type of the expressions depends on the identifier, constant or string.

\subsection{Unary Operators}
\label{ssec:unaryoperators}
There are two unary operators in \sys{}, and they are only used with a
numeric-valued operand such as a numeric constant
(see~\ref{sssec:numericconstants}).
These expressions are grouped right-to-left:
\startsyn
\texttt{+}\emph{numeric-expression} \\
\texttt{-}\emph{numeric-expression}
\stopsyn

\subsubsection{\texttt{+} operator}
This operator forces the value of its numeric operand to be positive.
The resulting expression is of numeric type with a value equal to the
absolute value of the numeric operand.

\subsubsection{\texttt{-} operator}
This operator forces the value of its numeric operand to be negative.
The resulting expression is of numeric type with a value equal to the
negative of the numeric operand.

\subsection{Channel/Calc Expresions}
\label{ssec:channelexpressions}
\texttt{Channel} and \texttt{Calc} types are the basis of \texttt{Image} and
\texttt{Kernel} objects respectively. There are several operators that
manipulate \texttt{Channel}s and \texttt{Calc}s.

\subsubsection{\texttt{:} operator}
\label{sssec:colonop}
Extract or use an individual \texttt{Channel} in an image.
\startsyn
\emph{image-identifier}\texttt{:}\emph{channel-identifier}
\stopsyn
The resulting expression has a type corresponding to the
extracted \texttt{Channel}.

\subsubsection{\texttt{\$()} operator}
\label{sssec:evalop}
This operator forces the evaluation of a previously defined \texttt{Image}
\texttt{Channel}. It is generally used in the context of a convolution operation.
\startsyn
\texttt{\$(}\emph{channel-expression}\texttt{)}
\stopsyn
The resulting expression has a type corresponding to the
calculated \texttt{Channel}.

\subsection{Composition Operators}
\label{ssec:compositionops}
These operators compose an \texttt{Image} from one or more \texttt{Channels}.
All channel composition operators are left-to-right associative.

\subsubsection{\texttt{|} operator}
\label{sssec:barop}
Compose two (or more) \texttt{Channel}s or \texttt{Calc}s. The resulting expression is a
\emph{multi-channel-expression}, or \emph{multi-calc} expression, and can be assigned
to either an \texttt{Image} or a \texttt{Kernel} object respectively.
\startsyn
\emph{channel-expression} \texttt{|} \emph{channel-expression} \\
\emph{multi-channel-expression} \texttt{|} \emph{channel-expression} \\
\emph{calc-expression} \texttt{|} \emph{calc-expression} \\
\emph{multi-calc-expression} \texttt{|} \emph{calc-expression} \\
\stopsyn
Note that \texttt{Channel}s and \texttt{Calc}s are appended in order, and
subsequent operations may rely on this order.

\comment{
\subsubsection{\texttt{||} operator}
\label{sssec:doublebarop}
Compose two (or more) \texttt{Channel}s. The resulting expression is a
\emph{multi-channel-expression}, and can be assigned to either an \texttt{Image}
or a \texttt{Kernel} object. This operator differs from the \texttt{|} operator in that
it forces the serial computation of channel values. This allows subsequent channel
value calculations to use neighboring pixels of previously calculated channels.
\startsyn
\emph{channel-expression} \texttt{||} \emph{channel-expression} \\
\emph{multi-channel-expression} \texttt{||} \emph{channel-expression}
\stopsyn
} % END comment (save this for later...)

\subsection{\texttt{**} operator}
\label{ssec:convolutionop}
MISSING: Talk about the core of our language\ldots the convolution operator

\subsection{Escaped ``C'' Expression}
\label{ssec:escapedC}
MISSING: Talk about the \texttt{\#[\ldots]\#} operator.

\subsection{I/O Expressions}

\subsubsection{\texttt{imgread} expression}
\label{sssec:imgread}
The \texttt{imgread} expression reads in an \texttt{Image} object from
a known image format located on the file system. The expression results
in an \texttt{Image} object which can be assigned using the \texttt{=}
operator (see section~\ref{sssec:equalop}). The resulting \texttt{Image}
object has 3 \texttt{Channel}s named \emph{Red}, \emph{Green}, and
\emph{Blue}. Each of the channels correspond to the red, green, and blue
image data read into the \texttt{Image} object. This expression is invoked
as a ``C'' style function, and expects 1 parameter: the path of the image
file to read.
\startsyn
\texttt{imgread(} \emph{string-constant} \texttt{)}
\stopsyn

\subsubsection{\texttt{imgwrite} expression}
\label{sssec:imgwrite}
The \texttt{imgwrite} expression writes out an \texttt{Image} object to a known
image format. It requires that the \texttt{Image} object has at least 3 named
\texttt{Channels}: \emph{Red}, \emph{Green}, and \emph{Blue}.
This expression has no type (null type), and is invoked as a ``C'' style function.
It expects 3 parameters: the first parameter is an \texttt{Image} identifier, the
second is the image format, and the the third is the path to which the image
should be written.
\startsyn
\texttt{imgwrite(} \emph{image-identifier} \texttt{,} \emph{string-constant} \texttt{,} \emph{string-constant} \texttt{)}
\stopsyn

\subsection{Assignmet Expresions}
\label{ssec:assignment}

\subsubsection{\texttt{=} assignment operator}
\label{sssec:equalop}
Assigns the value of the right operand to the left operand, copying data as necessary.
The types of both operands must match.

\subsubsection{\texttt{:=} assignment operator}
\label{sssec:colonequalop}
Assigns a calculation constant (see section~\ref{sssec:calcconstants}), or
escaped ``C'' expression (see section~\ref{ssec:escapedC}) to a \texttt{Calc}
object.

\subsubsection{\texttt{|=} assignment operator}
\label{sssec:barequalop}
Add a \texttt{Channel} or a \texttt{Calc} object to an \texttt{Image} or
\texttt{Kernel} object. Assignments using this operator are ordered by statement
order, and subsequent operations can rely on this order.

Note that a \texttt{Calc} object assigned to an \texttt{Image} object must be
evaluated using the \texttt{\$()} operator (see section~\ref{sssec:evalop}) before
being using in calculation or further assignment.

\comment{
\subsubsection{\texttt{\^=} assignment operator}
} % END comment - save for later...
