Learned in detail how the AST, scanner, parser and verifier are worked together to translate a programming language into an executable. Under some circumstances, it’s not as easy as I thought. For example, one needs to take the precedence and data types check into consideration from very beginning, such as designing a syntax tree. The way in which one part is designed may have significant influence to other parts. When we testing and fixing bugs in syntax, after a lot of syntax errors involving data types were discovered, we rewrote the syntax tree to be strongly typed instead of directly go to modify the parser and verifier, and that saved us a lot of time.

Also I learned how to design unit tests to discover all kinds of dark corners of a language and how to fix them. One cannot follow the routine when designing unit tests. Unusual conditions also need to be considered since they are always useful to detect bugs that out of one’s expectation. By retrieving back to the source code with results of other relative tests, the error is located and fixed.
