\chapter{Test Plan}
\label{chap:testplan}

Our testing is divided into four sections: syntax verification, semantic/type verification, CString verification, and functional output verification (image processing results comparison). 

\section{Syntax Verification}
\label{testing:syntax}

\section{Semantic Verification}
\label{testing:semantic}

\section{CString Verification}
\label{testing:cstrings}

\section{Image Processing Verification}
\label{test:output}


We start from part 1, using small unit tests to find out bugs in the syntax of CLAM. We carefully went through those data types and operators defined in CLAM to ensure the convergence. In detail, the default value of each basic type, the validity operands of all kinds of operators, and different ways of declaration are main points in this part. We also plan to fix those syntax bugs, especially in the parser and the verifier, immediately after them were found. Thus, at the end of the part 1, our  parser become more robust and capable of handling syntax errors.

In part two, we test the acceptance of wrapped C expression. The code in CLAM is translated into C language before being compiled, and C language expressions are supported during variable declaration. As a reason of that,  the parser needs to detect syntax errors in those wrapped C statements that may affect the compiling. Some tests about unreasonable statements are also involved such as 0 being denominator, or unclosed comment.

The functionalities of built-in functions and calculation operators in CLAM are tested in part 3. The actual images are required in the tests as input and output. The output of each test is observed and compared with desired result.

CLAM Syntax:
	Assignments: 
		image_eq_image.clam, image_oreq_image.clam
		image_oreq_image2.clam, image_defeq.clam 
		addker2img.clam, addker.clam, DefEq.clam

	** operator:
		convoperand.clam

	Variable deceleration:
		1calc_ker.clam, zerocalc1.clam, zerocalc2.clam
		matrix1.clam, matrix2.clam, matrix3.clam, matrix4.clam 
		defcalc1.clam, defcalc2.clam, defcalc3.clam
		keyword_id.clam,  id_overlap.clam, invalid_identifier.clam
		undefined_atomtype.clam
		

	Image operations syntax (not testing actual implementation):
		imgchannel1.clam, imgchannel2.clam, imgchannel3.clam, imgchannel4.clam
		imgread_bad.clam, imgwrite_bad1.clam, imgread_bad2.clam, imgread_bad3.clam
		defchannels.clam, sizediff.clam, equality_trans.clam

	The @ operator: at_channel.clam
	Comment: comment1.clam

	rval_calc.clam, rval_matrix.clam, rval_chanref.clam, rval_conv.clam, 
	rval_cstr.clam, rval_image.clam, rval_imgread.clam, rval_kernel.clam


Warped C syntax:
	cstring1.clam, cstring2.clam, cstring3.clam, cstring4.clam, cstring5.clam, cstring6.clam
	cimage.clam
	ckernel.clam

Image functionality:
	imgwrite_norgb.clam,  imgcopy.clam,  imgread.clam
	sobel.clam (general test)
