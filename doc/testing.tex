\chapter{Test Plan}
\label{chap:testplan}

Our unit test framework consists of pairs of identically named files in the \texttt{clam/tests/} directory.
Each pair consists of a CLAM program with extension \texttt{.clam} and an executable shell script
with extension \texttt{.test}. The CLAM file contains the code to be tested, while the shell script
specifies how to test that code: whether to compile and run or only compile, whether the test is supposed to fail,
what the expected output should look like, and what command line arguments to pass.
The \texttt{-buildup.sh} sets up the testing environment and defines common procedures such as
\texttt{compile-it} and \texttt{run-it}. Furthermore, \texttt{all.test} runs all tests in the directory,
tallies successes and failures and outputs a summary at the end.\\

Our testing is divided into four sections: syntax verification, semantic/type verification, CString verification, and functional output verification (image processing results comparison). 

\section{Syntax Verification}
\label{testing:syntax}

Syntax verification testing is meant to confirm that the parser accepts all valid token strings,
and rejects all invalid ones as defined in our language reference manual.
We achieve this by inspecting \texttt{clam/parser.mly} and writing unit tests for
potentially problematic cases (many of the more straightforward rules were not deemed test-worthy).
All of these tests are only compiled and not executed. The testing process uncovered a number of errors
in matrix parsing and definition of kernel calculation lists.\\

\begin{itemize}

\item \texttt{matrix1.clam} tests that a simple matrix parses correctly, and should compile:
\lstinputlisting[language=CLAM]{../clam/tests/matrix1.clam}
(This originally failed because a scale factor was required, but now it is accepted.)

\item \texttt{matrix2.clam} tests that a matrix with scaling factor parses correctly, and should compile:
\lstinputlisting[language=CLAM]{../clam/tests/matrix2.clam}

\item \texttt{matrix3.clam} tests whether the parser catches ill-formed matrices, and should not compile:
\lstinputlisting[language=CLAM]{../clam/tests/matrix3.clam}
(This originally succeeded due to incorrect matrix parsing rules, but now it fails.)

\item \texttt{matrix4.clam} tests whether the parser catches another type of ill-formed matrix, and should also not compile:
\lstinputlisting[language=CLAM]{../clam/tests/matrix4.clam}
(This originally succeeded due to incorrect matrix parsing rules, but now it fails.)

\item \texttt{keyword-id.clam} tests whether the parser allows keywords to be identifiers, and should not compile:
\lstinputlisting[language=CLAM]{../clam/tests/keyword-id.clam}

\item \texttt{1calc-ker.clam} tests whether the parser allows \texttt{Kernel} definitions with only one \texttt{Calc}, and should compile:
\lstinputlisting[language=CLAM]{../clam/tests/keyword-id.clam}
(This originally failed because the original syntax caused reduce/reduce errors, so we changed both the parser and the test.
There was originally no \texttt{|} after the \texttt{=}.)

\item \texttt{convoperand.clam} tests whether the \texttt{**} enforces a strict order of operands, and should not compile:
\lstinputlisting[language=CLAM]{../clam/tests/convoperand.clam}
(We only allow convolutions of the form \emph{channel-ref} \texttt{**} \emph{identifier}, in order to simplify
to translation into C code.)

\item \texttt{defcalc1.clam}, \texttt{defcalc2.clam}, and \texttt{defcalc3.clam} test various ways of declaring \texttt{Calc}s,
and all three should compile:
\lstinputlisting[language=CLAM]{../clam/tests/defcalc1.clam}
\lstinputlisting[language=CLAM]{../clam/tests/defcalc2.clam}
\lstinputlisting[language=CLAM]{../clam/tests/defcalc3.clam}

\item \texttt{rval-calc.clam}, \texttt{rval-matrix.clam}, \texttt{rval-chanref.clam}, \texttt{rval-conv.clam},
\texttt{rval-cstr.clam}, \texttt{rval-image.clam}, \texttt{rval-imgread.clam}, and \texttt{rval-kernel.clam}
test various "do nothing" statements consisting solely of r-values. Theses should all compile, though their
return values of the last line of each test are discarded so they do nothing:
\lstinputlisting[language=CLAM]{../clam/tests/rval-calc.clam}
\lstinputlisting[language=CLAM]{../clam/tests/rval-matrix.clam}
\lstinputlisting[language=CLAM]{../clam/tests/rval-chanref.clam}
\lstinputlisting[language=CLAM]{../clam/tests/rval-conv.clam}
\lstinputlisting[language=CLAM]{../clam/tests/rval-cstr.clam}
\lstinputlisting[language=CLAM]{../clam/tests/rval-image.clam}
\lstinputlisting[language=CLAM]{../clam/tests/rval-imgread.clam}
\lstinputlisting[language=CLAM]{../clam/tests/rval-kernel.clam}
(The parser accepted all of these, though most originally caused errors in the C translator
that had to be fixed (see "C Compiler Verification" below).)

\item \texttt{equality-trans.clam} checks that assignment expressions can be nested. This should compile:
\lstinputlisting[language=CLAM]{../clam/tests/equality-trans.clam}

\item \texttt{comment1.clam} checks that a program with only a comment (and zero statements) compiles. This should compile:
\lstinputlisting[language=CLAM]{../clam/tests/comment1.clam}

\end{itemize}

\section{Semantic Verification}
\label{testing:semantic}

Semantic verification testing is meant to confirm that the verifier accepts all valid parse trees,
and rejects all invalid ones according to the specifications of our language
(and according to what makes sense).
These tests depend on \texttt{clam/verifier.ml}, as well as \texttt{clam/environ.ml} and \texttt{envtypes.mli}. 
The testing process uncovered a number of errors in matrix verification (separate from syntax)
and the creation of default RGB Channels.\\

\begin{itemize}

\item\texttt{zerocalc1.clam} and \texttt{zerocalc2.clam} check that a matrix scaling-factor can
have a numerator of zero, but not a denominator of zero.
The former should certainly compile, while the latter should not:
\lstinputlisting[language=CLAM]{../clam/tests/zerocalc1.clam}
\lstinputlisting[language=CLAM]{../clam/tests/zerocalc2.clam}
(\texttt{zerocalc2.clam} originally passed verification, but that was fixed.)

\item\texttt{invalid1.clam} checks that undeclared identifiers are caught. This should not compile:
\lstinputlisting[language=CLAM]{../clam/tests/invalid1.clam}

\item\texttt{undefined1.clam} checks that an undefined Channel reference cannot be an r-value. This should not compile:
\lstinputlisting[language=CLAM]{../clam/tests/undefined1.clam}

\item\texttt{imgchannel1.clam} checks that an \texttt{Image} has default channels when read. This should compile:
\lstinputlisting[language=CLAM]{../clam/tests/imgchannel1.clam}

\item\texttt{imgchannel2.clam} checks that an undefined \texttt{Calc} cannot be applied to an \texttt{Image}. This should not compile:
\lstinputlisting[language=CLAM]{../clam/tests/imgchannel2.clam}

\item\texttt{imgchannel3.clam} checks that a \texttt{Calc} defined as a matrix cannot be applied to an \texttt{Image}. This should not compile:
\lstinputlisting[language=CLAM]{../clam/tests/imgchannel3.clam}

\item\texttt{imgchannel4.clam} checks that a \texttt{Calc} defined as a CString \emph{can} be applied to an \texttt{Image}. This should compile:
\lstinputlisting[language=CLAM]{../clam/tests/imgchannel4.clam}

\item\texttt{image-eq-image.clam} checks \texttt{=} assignment for images to images. This should compile:
\lstinputlisting[language=CLAM]{../clam/tests/image-eq-image.clam}

\item\texttt{image-oreq-image.clam} checks \texttt{|=} assignment for images to images. This is not supported and doesn't compile:
\lstinputlisting[language=CLAM]{../clam/tests/image-oreq-image.clam}

\item\texttt{image-defeq.clam} checks \texttt{:=} assignment for images. This is not supported and doesn't compile:
\lstinputlisting[language=CLAM]{../clam/tests/image-defeq.clam}

\item\texttt{imgread-bad.clam} checks that l-value identifiers have to be declared first. This shouldn't compile:
\lstinputlisting[language=CLAM]{../clam/tests/imgread-bad.clam}

\item\texttt{imgread-bad2.clam} checks that \texttt{imgread} must be called with only one argument. This shouldn't compile:
\lstinputlisting[language=CLAM]{../clam/tests/imgread-bad2.clam}

\item\texttt{imgread-bad3.clam} checks that \texttt{imgread} can only be called with a literal string or integer. This shouldn't compile:
\lstinputlisting[language=CLAM]{../clam/tests/imgread-bad3.clam}

\item\texttt{imgread.clam} checks that \texttt{imgread} can actually called with a correct argument. This should compile:
\lstinputlisting[language=CLAM]{../clam/tests/imgread-bad.clam}

\item\texttt{imgwrite-bad1.clam} checks that \texttt{imgwrite} can't be called with incorrect number and type of arguments. This shouldn't compile:
\lstinputlisting[language=CLAM]{../clam/tests/imgwrite-bad1.clam}

\item\texttt{defchannels.clam} checks that the default RGB Channels don't exist for a declared-but-not-read \texttt{Image}. This shouldn't compile:
\lstinputlisting[language=CLAM]{../clam/tests/defchannels.clam}
(Allowing default RGB channels for an unread \texttt{Image} is problematic because CLAM
does not allow explicit manipulation of Channel \emph{size}.)

\item\texttt{imgwrite-norgb.clam} checks that a declared-but-not-read \texttt{Image} without RGB channels
(and more importantly, without a \emph{size}) cannot be written. This shouldn't compile:
\lstinputlisting[language=CLAM]{../clam/tests/imgwrite-norgb.clam}

\item\texttt{sizediff.clam} checks that the default Channels from \texttt{Image}s of different sizes
cannot be assigned to each other, because all Channels of an \texttt{Image} should have the same size. This shouldn't compile:
\lstinputlisting[language=CLAM]{../clam/tests/sizediff.clam}

\item\texttt{at-channel.clam} checks that transient \texttt{Calc}s marked with \texttt{@} in a \texttt{Kernel}
do not result in Channels after a convolution. This shouldn't compile:
\lstinputlisting[language=CLAM]{../clam/tests/at-channel.clam}

\item\texttt{addker.clam} checks a \texttt{Kernel} can be appended to another \texttt{Kernel} using \texttt{|=}. This should compile:
\lstinputlisting[language=CLAM]{../clam/tests/addker.clam}

\item\texttt{DefEq.clam} checks that \texttt{=} cannot be used to define a \texttt{Calc}. This should not compile:
\lstinputlisting[language=CLAM]{../clam/tests/DefEq.clam}

\item\texttt{ckernel.clam} checks that \texttt{Kernel}s must be defined using \texttt{Calc} \emph{identifiers}
and not anonymous \texttt{Calc} expressions. This should not compile:
\lstinputlisting[language=CLAM]{../clam/tests/ckernel.clam}

\item\texttt{cimage.clam} checks that \texttt{Image}s cannot be defined using \texttt{Calc} expressions. This should not compile:
\lstinputlisting[language=CLAM]{../clam/tests/cimage.clam}

\end{itemize}

\section{CString Verification}
\label{testing:cstrings}

CString verification testing checks that invalid CStrings are caught before being passed to GCC, where they
could cause fatal errors. Of course, it is impossible to catch \emph{all} possible errors without
building an understanding of C syntax directly into CLAM, so we are satisfied with catching common
mistakes and rely on the user and GCC error messages for everything else. (For example, a CString of the form \texttt{\#[while(1)]\#},
while certainly disruptive, will compile and it is the user's fault that the program stalls.)
For each possible issue that we \emph{did} choose to catch, we wrote unit tests to confirm that they were caught.
Catching of invalid CStrings is done entirely in \texttt{clam/scanner.mll}. Testing reveals errors
in parentheses matching, which were promptly corrected.\\

\begin{itemize}

\item\texttt{cstring1.clam} checks that C preprocessor cannot be used in a CString. This should not compile:
\lstinputlisting[language=CLAM]{../clam/tests/cstring1.clam}

\item\texttt{cstring2.clam} checks that C comments cannot be used in a CString. This should not compile:
\lstinputlisting[language=CLAM]{../clam/tests/cstring2.clam}

\item\texttt{cstring3.clam} checks that empty CStrings are OK, and just result in an empty statement. This should compile:
\lstinputlisting[language=CLAM]{../clam/tests/cstring3.clam}

\item\texttt{cstring4.clam} checks that library calls in a CString must be closed. This should not compile:
\lstinputlisting[language=CLAM]{../clam/tests/cstring4.clam}
(This originally did compile because the scanner didn't record nesting-level properly. It was fixed.)

\item\texttt{cstring5.clam} checks that parentheses in a CString must be matched. This should not compile:
\lstinputlisting[language=CLAM]{../clam/tests/cstring5.clam}
(This originally did compile because the scanner didn't record nesting-level properly. It was fixed.)

\item\texttt{cstring6.clam} is a sanity check that "normal" CStrings do in fact work. This should obviously compile:
\lstinputlisting[language=CLAM]{../clam/tests/cstring6.clam}

\end{itemize}

\section{C Compiler Verification}
\label{test:ccompiler}

A few of the above tests, which originally compiled, began failing once the C backend was hooked up.
The most common errors were with C syntax and memory allocation.
While we didn't write tests specifically targeted at the C backend,
the tests we wrote for other parts of the architecture also helped us identify problems in the C translator. 

\section{Image Processing Verification}
\label{test:output}

These are comprehensive tests, consisting of full-fledged programs that actually
read, manipulate and write images. While the testing framework provides image-comparison functionality,
most verification here is actually done with the naked eye.\\

\begin{itemize}

\item\texttt{imgcopy.clam} copies an image. This should compile, run, and copy an image:
\lstinputlisting[language=CLAM]{../clam/tests/imgcopy.clam}

\item\texttt{sobel.clam} applies the Sobel operation to an image. This should compile, run, and output
a map of the luminosity gradient:
\lstinputlisting[language=CLAM]{../clam/tests/sobel.clam}

\end{itemize}


