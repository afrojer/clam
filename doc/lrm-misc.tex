\section{Program Definition}
A program in the \sys{} language is simply a sequence of statements which
are executed in order.

\section{Scope Rules}
All identifiers in the \sys{} language are global.
%Removed:
%, except for the
%identifiers prefixed with an \texttt{@} symbol, which can only be
%accessed by their own calculation.

In an escaped C block that defines a channel, the existing channels
for an image will be in scope when the block is executed. Because
this block will be executed on every pixel, the name of the channel
will bind to the current pixel value for that channel. These bindings
will be resolved when the channel is calculated; not when it is
defined.

\section{Declarations}
All variables must be declared before they can be used. However,
variable declarations can be made at any point in a program.
A variable becomes usable after the end of the semi-colon of the statement
in which it's contained.
