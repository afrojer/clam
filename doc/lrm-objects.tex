\section{Objects and Definitions}
\label{sec:objdef}
An \emph{object} in \sys{} is either a named collection of \texttt{Channel}s, called an
\texttt{Image}, or a named collection of calculation basis, called a
\texttt{Kernel}. A \texttt{Channel} is a mathematical matrix of numeric values
whose individual components are not directly accessible via \sys{} language
semantics -- \texttt{Channel} values are manipulated via the convolution
operator (see~\ref{ssec:convolutionop}). A calculation basis, known as a
\texttt{Calc}, is a collection of either calculation constants
(see~\ref{sssec:calcconstants}) or calculation expressions (see~\ref{ssec:escapedC}),
or both.

\subsection{\texttt{Image} objects}
\label{ssec:images}
An \texttt{Image} is a collection of named \texttt{Channel}s. \texttt{Channel}s can
be dynamically added \comment{or removed} using the channel composition
operator (see section~\ref{sssec:barequalop}, or by assigning to a previously
undeclared \texttt{Channel} name. 

For example, to create a gray-scale image from a single, pre-existing
\texttt{Channel}:
\begin{lstlisting}[language=CLAM,escapechar=\%]
Image outImg;
outImg:Red = calcImg:G;
outImg:Green = calcImg:G;
outImg:Blue = calcImg:G;
\end{lstlisting}

\subsection{\texttt{Kernel} objects}
\label{ssec:kernels}
A \texttt{Kernel} is an ordered collection of calculation basis which is used by the convolution
operator (see section~\ref{ssec:convolutionop}). Each calculation basis can be either
a calculation constant (see~\ref{sssec:calcconstants}) or a calculation expression
(see~\ref{ssec:escapedC}). A \texttt{Kernel} is composed either using the composition
operator (see section~\ref{sssec:barop}), or the \texttt{|=} assignment operator (see section~\ref{sssec:barequalop}).

To see how a \texttt{Kernel} is used in calculation, see section~\ref{ssec:convolutionop}.
