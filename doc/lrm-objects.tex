\section{Objects and Definitions}
\label{sec:objdef}
An \emph{object} in \sys{} is either a named collection of Channels, called an
\texttt{Image}, or a named collection of calculation bases, called a
\texttt{Kernel}. A Channel is a mathematical matrix of numeric values
whose individual components are not directly accessible via \sys{} language
semantics -- Channel values are manipulated via the convolution
operator (see~\ref{ssec:convolutionop}). A calculation basis, known as a
\texttt{Calc}, is either a calculation constant
(see~\ref{sssec:calcconstants}) or a calculation expression (see~\ref{ssec:escapedC}).

\subsection{\texttt{Calc} objects}
\label{ssec:calc}
A  \texttt{Calc} object is an immutable object initialized with the \texttt{:=} assignment operator
(see section~\ref{sssec:colonequalop}). Its assigned value is either a CString (section~\ref{ssec:escapedC}),
or a calculation constant (\emph{matrix} - section~\ref{sssec:calcconstants}). The calculation
described by a \texttt{Calc} object will be performed for each pixel in an \texttt{Image} Channel.
The calculation is performed using either the convolution operator (section~\ref{sssec:convolutionop}),
or the channel composition operator (section~\ref{sssec:barequalop}).

\subsection{\texttt{Image} objects}
\label{ssec:images}
An \texttt{Image} is a collection of named Channels. Channels can
be dynamically added \comment{or removed} using the channel composition
operator (see section~\ref{sssec:barequalop}, or by assigning to a previously
undeclared Channel name. 

For example, to create a gray-scale image from a single, pre-existing
Channel:
\begin{lstlisting}[language=CLAM,escapechar=\%]
Image outImg;
outImg:Red = calcImg:G;
outImg:Green = calcImg:G;
outImg:Blue = calcImg:G;
\end{lstlisting}

\subsection{\texttt{Kernel} objects}
\label{ssec:kernels}
A \texttt{Kernel} is an ordered collection of calculation bases which is used by the convolution
operator (see section~\ref{ssec:convolutionop}). Each calculation must to be identified with a \texttt{Calc}
identifier, but the underlying basis can be either
a matrix calculation constant (see~\ref{sssec:calcconstants}) or an escaped C expression
(see~\ref{ssec:escapedC}). A \texttt{Kernel} is composed either using the composition
operator (see section~\ref{sssec:barop}), or the \texttt{|=} assignment operator (see section~\ref{sssec:barequalop}).

To see how a \texttt{Kernel} is used in calculation, see section~\ref{ssec:convolutionop}.
