\section{Lexical Conventions}
\label{sec:lex}

\subsection{Tokens}
\label{ssec:tokens}
The tokens in \sys{} are broken down as follows: We have reserved
keywords, identifiers, constants, control characters, and operators.
The end of a token is defined by the presence of a newline, space,
tab character (whitespace), or other character that
cannot possibly be part of the current token.

\subsection{Comments}
\label{ssec:comments}
Comments are demarcated with an opening /* and closing */, as in C.
Any characters inside the comment boundaries are ignored. Comments
can be nested.

\subsection{Keywords}
\label{ssec:keywords}
The reserved keywords in \sys{} are:
\begin{center}\begin{tabular}{l l l l}
Image & imgread & Int8 & Uint8\\
Kernel & imgwrite & Int16 & Uint16\\
Calc & Angle & Int32 & Uint32
\end{tabular}\end{center}

\subsection{Identifiers}
\label{ssec:identifiers}
Identifiers are composed of an upper or lower-case letter immediately
followed by any number of additional letters and/or digits. Identifiers
are case sensitive, so ``foo'' and ``Foo'' are different identifiers.
Identifiers cannot be keywords, and underscores are disallowed.

\subsection{Constants}
\label{ssec:constants}
In \sys{} there are 3 types of constants: numeric constants,
calculation constants, and string literals.

\subsubsection{Numeric Constants}
\label{sssec:numericconstants}

\emph{Integers} are repesented by a series of number characters.
% the negative sign is dealt with by the unary '-' operator

Angles are represented by a series of number characters with an
optional period character.

\subsubsection{Calculation Constants}
\label{sssec:calcconstants}
Matrix calculation constants are represented by an opening curly brace, followed
by a series of \emph{numeric-expressions} separated by whitespace or
comma characters. The comma  characters represents the division between
the rows of the matrix. Each row must have the same number of
\emph{numeric-expressions}, but the matrix need not be square.

A calculation constant may also have an optional fraction preceding it,
which indicates that every value in the matrix should be multiplied
by that fraction. The fraction will be expressed as an opening
bracket character, a \emph{numeric-expression} representing the
numerator, a forward-slash character, a \emph{numeric-expression}
representing the denominator, and a closing bracket character.

\startsyn
\texttt{\{} \emph{numeric-expr} \emph{numeric-expr} \ldots \texttt{,} \emph{numeric-expr} \emph{numeric-expr}\ldots \texttt{\}} \\
\texttt{[}\emph{numeric-expr} \texttt{/} \emph{numeric-expr} \texttt{]}\texttt{\{} \emph{numeric-expr} \emph{numeric-expr} \ldots \texttt{,} \emph{numeric-expr} \emph{numeric-expr}\ldots \texttt{\}}
\stopsyn

The following is an example of a calculation constant.
\begin{lstlisting}[language=CLAM,escapechar=\%]
Calc sobelGy := [1 / 9]{1 3 1 , 2 -5 2 , 1 3 1 };
\end{lstlisting}

\subsubsection{String Literals}
\label{sssec:strings}

String constants are demarcated by double quote characters or single
quote characters. Consecutive string constants will be automatically
appended together into a single string constant.
String constants may contain escaped characters, but since Strings
are only used by for filenames this is of limited utility.
\startsyn
\texttt{"}\emph{string-constant}\texttt{"}
\stopsyn
