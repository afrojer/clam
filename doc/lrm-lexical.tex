\section{Lexical Conventions}
\label{sec:lex}

\subsection{Tokens}
\label{ssec:tokens}
The tokens in \sys{} are broken down as follows: We have reserved
keywords, constants, control characters, identifiers, and operators.
The end of a token is defined by the presence of a newline, space,
or tab character (whitespace), or by the presence of a character that
cannot possibly be part of the current token.

\subsection{Comments}
\label{ssec:comments}
Comments are demarcated with an opening /* and closing */, as in C.
Any characters inside the comment boundaries are ignored. Comments
can be nested.

\subsection{Identifiers}
\label{ssec:identifiers}
Identifiers are composed of an upper or lower-case letter immediately
followed by any number of additional letters and/or digits. Identifiers
are case sensitive, so ''foo`` and ''Foo`` are different identifiers.
Identifiers cannot be keyworeds, and underscores are disallowed.

\subsection{Keywords}
\label{ssec:keywords}
The reserved keywords in \sys{} are:
Import, Export, Kernel, Channel, 
Int8, Int32, Angle, and Float.


\subsection{Constants}
\label{ssec:constants}
In \sys{} there are 5 types of constants: string constants, matrix
constants, integers, angles, and floats.

\subsubsection{Numeric Constants}
\label{sssec:numericconstants}

Integers are repesented by a series of number characters, with an
optional negative sign in front.

Angles are represented by a series of number characters with an
optional period character, followed by a lower-case ''a``.

Floats are represented by a series of number characters with an
optional period character, followed by a lower-case ''f``.

\subsubsection{Matrix Constants}
\label{sssec:matrixconstants}
Matrix constants are represented by an opening curly brace, followed
by a series of integers separated by whitespace or pipe characters.
The pipe characters represent the division between the rows of the
matrix. Each row must have the same number of integers, but the
matrix need not be square.

A matrix constant may also have an optional fraction preceding it,
which indicates that every value in the matrix should be multiplied
by that fraction. The fraction will be expressed as an opening
bracket character, an integer representing the numerator, a pipe
character, an integer representing the denominator, and a closing
bracket character.

The following is an example of a matrix constant.
\begin{lstlisting}[language=CLAM,escapechar=\%]
    Matrix sobelGy := [N | D]{R1C1 R1C2%\ldots% | R2C1 R2C2%\ldots% | R3C1 R3C2%\ldots%};
\end{lstlisting}

\subsection{String Literals}
\label{ssec:strings}

String constants are demarcated by double quote characters or single
quote characters. Consecutive string constants will be automatically
appended together into a single string constant.

