\chapter{Language Tutorial}

\section{Input and Output}

\sys{}'s basic I/O operators are \texttt{imgread} and \texttt{imgwrite}, and every program
will have to call them at least once each to do anything useful. \texttt{imgread}
takes a filename (or integer, see below) as its sole argument and returns an \texttt{Image}.
\texttt{imgwrite} take an \texttt{Image}, a format, and filename (or integer).

\subsection{Your first program}

Using only I/O operators, you can already write a simple program that copies an image from
one location to another. Or, if the output is in a different format than the input,
you have a simple image converter. In either case, you only need two lines of code!

\begin{lstlisting}[language=CLAM,escapechar=\%]
Image input = imgread("source.jpg");
imgwrite(input, "png", "dest.png");
\end{lstlisting}
\ldots or 1 if you're tricky
\begin{lstlisting}[language=CLAM,escapechar=\%]
imgwrite(imgread("source.jpg"), "png", "dest.png");
\end{lstlisting}

\subsection{Using command-line arguments}

This program only copies from \texttt{"source.jpg"} to \texttt{"dest.png"} which is not
very useful.
You'd need to edit the code and recompile to change the source and destination.
To avoid this problem, \texttt{imgread} and \texttt{imgwrite} can both be called with integers
which refer to items in the command-line argument list, giving your code much greater
flexibility. (\sys{} will automatically enforce the correct number of command-line arguments.)

\begin{lstlisting}[language=CLAM,escapechar=\%]
imgread(1); /* Get filename from argv[1]*/
imgwrite(input, "png", 2); /* Get filename from argv[2]*/
\end{lstlisting}

\section{Compiling and Running Your Program}

In order to generate binaries from your \sys{} program, you will need the \texttt{g++} compiler
installed and available in your default \texttt{PATH}. This is because \sys{} uses C/C++ as its
compile target, and leverages existing C/C++ compilers to generate and optimize machine-dependent code.
You can compile your clam program by simply passing your file to \texttt{clam} as the sole argument.
This will automatically output a binary to \texttt{a.out}. You can also specify the name of
the binary with the \texttt{-o} flag, and pass in the source file path with \texttt{-i}.
For example:
\begin{itemize}
  \item \texttt{./clam prog1.clam}
  \item \texttt{./a.out source.jpg dest.png}
  \item \texttt{./clam -i prog1.clam -o copyimg}
  \item \texttt{./copyimg source.jpg dest.png}
\end{itemize}

The full set of options to the \sys{} compiler can be found using the \emph{--help} option, and is
reproduced below:
\begin{lstlisting}[language=make,basicstyle=\ttfamily]
CLAM v0.1
Usage: clam {options} [{<} inputfile]
Options are:
  -o <filename> Specify the output file
  -i <filename> Specify the input file
  -c            Output generated C only
  -t            Print AST debugging information
  -help         Display this list of options
  --help        Display this list of options
\end{lstlisting}

\section{Basic Types}

\subsection{Channels}

\emph{Channels} are arrays of values associated each pixel in an \texttt{Image}.
For example, each pixel in an \texttt{Image} usually has \emph{Red}, \emph{Green}, and \emph{Blue} values
associated with it, and we can further define \emph{Luminosity}, \emph{Hue}, \emph{Saturation} etc.
Thus we can refer to the \emph{Red} channel or the \emph{Saturation} channel of an \texttt{Image}.

When first read into \sys{}, \texttt{Image}s come with three default Channels:
\texttt{Red}, \texttt{Green}, and \texttt{Blue}. These can be accessed using the \texttt{:} (colon) operator.
The values in one Channel can be copied to another using the \texttt{=} (equals) operator.
If the Channel on the left-hand side is undefined, it is created dynamically.

The following program uses a \texttt{temp} channel to swap the \texttt{Red} and \texttt{Blue} values of an image:

\begin{lstlisting}[language=CLAM,escapechar=\%]
Image img1 = imgread(1);

img1:temp = img1:Blue;
img1:Blue = img1:Red;
img1:Red = img1:temp; /* swap channels */

/*Only Red, Green, and Blue channels are written:*/
imgwrite(img1, "jpg" ,2);
\end{lstlisting}

\subsection{Calculations}

While the equals operator is enough to create new channels that are copies of old ones,
\texttt{Calc} objects allow you to create new Channels via calculation.
The \texttt{:=} (colon-equals) operator is used to define \texttt{Calc}s object.
Once defined, a \texttt{Calc} object cannot be redefined.
A \texttt{Calc} can be assigned an \emph{atomic type} such as \texttt{<Uint16>} or \texttt{<Angle>},
and all values resulting from that calculation will be clamped to the appropriate range.
The default type is \texttt{<Uint8>}, which corresponds to a range of 0-255.
  
\texttt{Calc} objects can be defined in two ways - as \emph{escaped-C strings} (\emph{CStrings}) or as \emph{matrices}.

CString \texttt{Calc}s are enclosed in \texttt{\#[\ldots]\#} brackets. These strings can contain basic mathematical
operators and functions, as well as references to other Channels. A \texttt{Calc}
defined in this manner can be applied to an \texttt{Image} using the \texttt{|=} (or-equal) operator,
provided that the \texttt{Image} has all the requisite Channels,
thereby creating a new Channel with the same name as the \texttt{Calc}. The values of this new
Channel will be calculated according
to the contents of the string. (It follows that anonymous \texttt{Calc}s are not allowed.)\\

\begin{lstlisting}[language=CLAM,escapechar=\%]
/* Define a calculation for Luminosity */
Calc Lum<Uint8> := #[(3*Red + 6*Green + 1*Blue) / 10]#;
Calc Zero := #[0]#;

Image srcimg = imgread(1);
/* Add luminosity channel to the Image */
srcimg |= Lum;

/* Add a 'black' channel to the Image, named 'Zero' */
srcimg |= Zero;

/* The following is invalid - no name! */
srcimg |= #[Red + Green + Blue]#;

/* Calcs cannot be redefined! */
Lum := #[Red * Green * Blue]#
\end{lstlisting}

\emph{Matrix} \texttt{Calc}s can be of any size, and are represented as lists of numbers enclosed in
\texttt{\{\ldots\}} braces. Rows are separated by commas, and the \emph{Matrix} is optionally preceded by
a scaling factor of the form \texttt{[} \emph{numerator} \texttt{/} \emph{denominator} \texttt{]}.
Matrices represent a weighted (and scaled) sum of values in the neighborhood of a pixel.
Because matrix \texttt{Calc}s cannot be calculated with a simple ``for loop,'' over all pixels,
they cannot be applied
directly to \texttt{Image}s. However they can be added to \texttt{Kernels} and then \emph{convolved}
with \texttt{Image}s (see section~\ref{sec:tutorial:kernels}) and are useful in a wide range of applications.\\

\begin{lstlisting}[language=CLAM,escapechar=\%]
Calc Avg<Uint8> := [1/9] { 1 1 1, 1 1 1, 1 1 1 };
/* This matrix averages the values in a 3x3 square
	centered on a given pixel */

/* This doesn't work: */
srcimg |= Avg;
\end{lstlisting}

\subsection{Kernels}\label{sec:tutorial:kernels}

\texttt{Kernel} are ordered collections of \texttt{Calc}s. They are defined with the \texttt{=} (equals) operator
and a list of \texttt{Calc}s separated by the | (or) operator. More \texttt{Calc}s can be added to
a \texttt{Kernel} afterwards using the \texttt{|=} (or-equal) operator. A \texttt{Calc} in a \texttt{Kernel}
can be prefixed with the \texttt{@} (at) symbol to indicate that it is an intermediate calculation
(see section~\ref{sec:tutorial:convolutions}).

\begin{lstlisting}[language=CLAM,escapechar=\%]
Calc sobelGx<Uint8> := {-1 0 +1, -2 0 +2, -1 0 +1};
Calc sobelGy<Uint8> := {+1 +2 +1, 0 0 0, -1 -2 -1};
Calc sobelG<Uint8> :=
    #[sqrt(sobelGx * sobelGx + sobelGy * sobelGy)]#;
Kernel k = | @sobelGx | @sobelGy | sobelG;
/* Calcs can refer to preceding Calcs in same kernel */

Calc sobelTheta := #[atan((float)sobelGx/(float)sobelGy)]#;
k |= sobelTheta; 
/* don't have to add all Calcs at once */
\end{lstlisting}

\section{Convolutions}\label{sec:tutorial:convolutions}

The \texttt{**} operator takes a Channel reference and a \texttt{Kernel}, and
applies the \texttt{Kernel}'s \texttt{Calc}s in sequence to the Channel.
Matrices are applied to the directly to the \texttt{Image} Channel specified,
while CStrings generally calculate a value using other Channels. Any Channel which has
already been calculated in the convolution may be used by a CString \texttt{Calc}.
The result of a convolution is an \texttt{Image} that has an initialized Channel corresponding to
each \texttt{Calc} defined in the \texttt{Kernel} which was \emph{not} marked as intermediate
(prefixed with \texttt{@} symbol).

Continuing the previous example, we can take the \texttt{Kernel}, \emph{sobel}, and apply it to an \texttt{Image}:

\begin{lstlisting}[language=CLAM,escapechar=\%]
Image edges = srcimg:Lum ** sobel;
/* edges:sobelG and edges:sobelTheta now valid */
/* but not edges:sobelGx or edges:sobelGy */
\end{lstlisting}

\clearpage
\section{Full Example}
\label{sec:tutorial:fullexample}

The last few examples have included portions of the \emph{Sobel} edge detecting operator.
While in most programming languages implementing the Sobel operator
is complicated and error-prone (with multiple nested loops), the \sys{} version is straightforward and
given in its entirety below:

\lstinputlisting[language=CLAM]{src/sobel.clam}

