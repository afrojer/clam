\chapter{Language Tutorial}

\section{Input and Output}
\begin{lstlisting}[language=CLAM,escapechar=\%]
Image my_img = imgread(<file-name>);

imgwrite(<image-identifier>, <img-type>, <file-name>);
\end{lstlisting}

Your first program (Image Copier/Converter):

\begin{lstlisting}[language=CLAM,escapechar=\%]
Image input = imgread("source.jpg");
imgwrite(input, "png", "dest.png");

Also can use command line arguments:
imgread(1); /* reads first  argument */
imgwrite(input, "png", 2); /* writes to second arguments */
\end{lstlisting}

\section{Running Your Program}

> ./clam program1.clam
Translates to C, compiles, outputs to a.out
> ./clam -i program1.clam -o out
Translates to C, compiles, outputs to "-o" file
> ./clam -c program1.clam
Translates to C, prints C code to clam_gen.c
> ./clam -t program1.clam
Debugging: Print abstract syntax tree

\section{Channels}

Channels are arrays of values for each pixel, such as Red, Green, and Blue.

Images come with these three default channels when read.
Access using ":" operator (img:Red, img:Green, img:Blue)
Set/create channels using "=" operator

\begin{lstlisting}[language=CLAM,escapechar=\%]
Image img1 = imgread("image.jpg");

img1:temp = img1:Blue;
img1:Blue = img1:Red;
img1:Red = img1:temp; /* swap channels */

/*Only Red, Green, and Blue channels are written:*/
imgwrite(img1,"jpg","image.jpg");
\end{lstlisting}

\section{Calculations}

Calculations, which are to be applied to each pixel,
can be defined in two ways: 
    matrices (containing weights for neighboring pixels)
    or C strings (containing references to the same pixel in other channels)

\begin{lstlisting}[language=CLAM,escapechar=\%]
Calc m<Uint8> := [1 / 9] {1 1 1, 1 1 1, 1 1 1};
Calc Lum := #[(3*Red + 6*Green + 1*Blue) / 10]#;
\end{lstlisting}

C string calculations can be added to Images,
creating new channels (with the same name as the Calc):
\begin{lstlisting}[language=CLAM,escapechar=\%]
srcimg |= Lum; /* srcimg:Lum is now valid */
\end{lstlisting}

The values of the Channel will be calculated on first use.

Calcs must have names! (defined once with ":="):
\begin{lstlisting}[language=CLAM,escapechar=\%]
srcimg |= #[Red + Green + Blue]#; /* INVALID */
\end{lstlisting}

Adding matrix calculations to Images is meaningless,
but they have other important uses...

\section{Kernels}

Kernels are ordered collections of calculations.

\begin{lstlisting}[language=CLAM,escapechar=\%]
Calc sobelGx<Uint8> := {-1 0 +1, -2 0 +2, -1 0 +1};
Calc sobelGy<Uint8> := {+1 +2 +1, 0 0 0, -1 -2 -1};
Calc sobelG<Uint8> := 
    #[sqrt(sobelGx * sobelGx + sobelGy * sobelGy)]#;
Kernel k = @sobelGx | @sobelGy | sobelG;
/* Calcs can refer to preceding Calcs in kernel */
/* "@" means generate value, but don't generate channel */
/* (this will make sense when we see convolutions) */

Calc sobelTheta := #[arctan(sobelGx/sobelGy)]#;
k |= sobelTheta; 
/* don't have to add all Calcs at once */
\end{lstlisting}

\section{Convolutions}

The "**" operator takes a Channel reference and a Kernel,
applies the calcs in sequence (matrices are applied to the specific Channel given),
and generates an Image with all the channels (Calcs) defined in that Kernel not prefixed with "@"

Continuing the previous example:

\begin{lstlisting}[language=CLAM,escapechar=\%]
Image edges = srcimg:Lum ** sobel;
/* edges:sobelG and edges:sobelTheta now valid */
/* but not edges:sobelGx or edges:sobelGy */
\end{lstlisting}

\section{Full Program (Sobel Operator)}

\lstinputlisting[language=CLAM]{src/sobel.clam} \clearpage


