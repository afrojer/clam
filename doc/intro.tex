\chapter{Introduction}

\sys{} is a linear algebra manipulation language specifically targeted for
image processing. It provides an efficient way to express complex image
manipulation algorithms through compact matrix operations. Traditional image
processing is performed using a language such as C, or C++. Algorithms in these
languages are quite complex and error-prone due to the large number of lines of
code required to implement something as conceptually simple as, "make this image
blurry." The complexity arises from the need to perform elaborate calculations
on every pixel in an image. For example, to blur an image you first need to
calculate the luminance of the pixel (from the red, green, and blue channels),
then you need to mathematically combine this with the luminance of adjacent pixels,
and finally re-calculate red, green, and blue values for an output image.

\sys{} will simplify image processing, and more generally linear algebra, through
domain-specific data types and operators. A basic data type in \sys{} is a 
\texttt{Calc}, which can either be a \texttt{Matrix} or a \texttt{CString}. 
Matrices can be variable sized with an optional rational coefficient, and are
used in image convolution operations. \texttt{CString}s are simple calcuations
based on previously defined channels and basic C math operations such as \texttt{sqrt}
or \texttt{atan}.

An \texttt{Image} is another \sys{} data type which is expressed as a collection
of channels. For example, when reading an image into memory, \sys{} creates a
\emph{Red}, \emph{Green}, and \emph{Blue} channel automatically. Additional
Image channels can either be assigned from other images, or calculated
using an expression syntax which defines a calculation involving the
values of other, previously defined, channels. The basic image processing operator
in \sys{} is the convolution operator. This operator takes a \texttt{Calc} and a
\texttt{Kernel}, another basic data type, and outputs an \texttt{Image}. This
operator convolves each \texttt{Matrix} within the \texttt{Kernel} with the input channel,
and collects the resulting output channels into a new \texttt{Image}.

Two primary use cases of \sys{} are basic image information extraction, and
filtering. The compact syntax and powerful basic data types of \sys{} will make
information extraction, such as finding all the edges in an image, simple, compact,
and easy to read.

